\documentclass[a4paper, 11pt]{article}
\usepackage[english]{babel}
\usepackage[top=2cm,bottom=2cm,left=2cm,right=2cm]{geometry}
\usepackage[utf8]{inputenc}
\usepackage{import}
\usepackage{float}
\usepackage{subfigure}
% \usepackage{subfig}
\usepackage[pdftex]{graphicx}
% \usepackage{graphicx}
\usepackage{amssymb,amsmath,amsthm,amsfonts}
\usepackage{xspace}
\usepackage{tabularx}
\usepackage{indentfirst}
\usepackage{wrapfig,booktabs}
%\usepackage[small]{caption}
% \usepackage{subcaption}
\usepackage{eucal}
\usepackage{eso-pic}
\usepackage{hyperref}
\usepackage{url}
\usepackage{booktabs}
\usepackage{afterpage}
\usepackage{parskip}
\usepackage{listings}
\usepackage{fancyhdr}
\usepackage{textcomp}
\usepackage{cite}
\usepackage{multirow,multicol}
\usepackage{setspace}
\usepackage[version=4]{mhchem}
\usepackage{nicefrac}
\usepackage{siunitx}

\usepackage{caption}
\captionsetup{tableposition=top,font=small,width=0.8\textwidth}
%\usepackage[table]{xcolor}
\usepackage[arrowdel]{physics}
\usepackage{mathtools}
\usepackage{tablefootnote}
\usepackage{enumitem}

\setlist[description]{font={\scshape}} %style=unboxed,style=nextline
\usepackage{floatflt}
\usepackage{commath}
\usepackage{bm}
\usepackage{ifthen}
\usepackage{comment}
\usepackage[colorinlistoftodos,textsize=tiny]{todonotes}

\newcommand{\overbar}[1]{\mkern 1.5mu\overline{\mkern-1.5mu#1\mkern-1.5mu}\mkern 1.5mu}
\let\oldfrac\frac
\renewcommand{\frac}[3][d]{\ifthenelse{\equal{#1}{d}}{\oldfrac{#2}{#3}}{\nicefrac{#2}{#3}}}


\begin{document}

\title{Study of the relaxation of the Al (100) surface via Ab Initio simulations}
\author{Alessandro Lovo, mat. 1236048}

\maketitle

\section{Introduction}
  The aim of this report is use Ab Initio (AI) methods to study the surface of an Aluminium crystal. In order to do this the Quantum Espresso \cite{rif:QE} programm will be used to first find the bulk value of the lattice constant for Al and then simulate a surface to study its relaxation from the bulk configuration.

\section{Estimation of the bulk lattice constant}
  The idea to find the equilibrium lattice constant $a_0$ is to simulate the crystal at different values of $a$ and for each of them compute the pressure $P$ resulting on the system. At this point data can be fitted with the Murnaghan equation of state:

  \begin{equation*}
    P(V;V_0,K_0,k_0') = \frac{K_0}{K_0'}\left(\left(\frac{V}{V_0} \right)^{-K_0'} - 1 \right)
  \end{equation*}

  where $V$ is the volume of the unit cell and $V_0$ the one at equilibrium, while the bulk modulus is $K = -V \left(\frac{\partial P}{\partial V}\right)_T \approx K_0 + K_0'P$.
  But to have meaningful results from the simulations one has first to optimize some parameters of the Quantum Espresso (QE) code.

  \subsection{Optimization of the simulation parameters}
    \paragraph{Structure of the unit cell}
      Since the crystalline structure of Al is face centerd cubic (fcc) with no basis, it is enough to consider an fcc unit cell with a single Al atom at the origin.
    \paragraph{Pseudopotential}
      From all the many pseudopotentials provided in the QE library, here a non relativistic one (\verb|Al.pz-vbc.UPF|) and a scalar-relativistic one (\verb|Al.pbe-nl-rrkjus_psl.1.0.0.UPF|) are tested.
      With the latter one obtains a better approximation but the simulations take longer.
    \paragraph{Number of k points and smearing}
      To sample the first Brillouin zone a grid with \verb|n k points| in each direction is used: the higher the number the more accurate is the result, but also the longer the simulation takes. To mitigate this coarse sampling of the Brillouin zone one can use a smearing of the energy levels whose amplitude is controlled by the variable \verb|degauss|. By monitoring how the total energy behaves with these two parameters one can find a proper value for them: $\verb|n k points| = 8$, $ \verb|degauss| = 0.02 \si{.Ry}$ (fig \ref{fig:bulk_degauss-n_k} where the two 1D plots are taken at the best value of the other parameter.).

    \begin{figure}
      \centering
      \resizebox{0.45\textwidth}{!}{\import{img/}{bulk_degauss-n_k_rel.pgf}}
      \resizebox{0.45\textwidth}{!}{\import{img/}{bulk_degauss-n_k_rel_time.pgf}} \\
      \resizebox{0.45\textwidth}{!}{\import{img/}{bulk_n_k_rel.pgf}}
      \resizebox{0.45\textwidth}{!}{\import{img/}{bulk_degauss_rel.pgf}}
      \caption{Behavior of the total energy and the simulation time as a function of the number of k points and of the amount of smearing. Simulations performed on the relativistic pseudopotential, but with the other the results are very similar.}
      \label{fig:bulk_degauss-n_k}
    \end{figure}

    % \begin{figure}[H]
    %   \centering
    %   \begin{subfigure}[]{
    %     \resizebox{0.45\textwidth}{!}{\import{img/}{bulk_n_k_rel.pgf}}
    %     \label{fig:bulk_n_k}}
    %   \end{subfigure}
    %   \begin{subfigure}[]{
    %     \resizebox{0.45\textwidth}{!}{\import{img/}{bulk_degauss_rel.pgf}}
    %     \label{fig:bulk_degauss}}
    %   \end{subfigure}
    % \end{figure}

  \paragraph{Energy cutoffs}
    The orbitals for the electron are expanded in plain waves, and the number of waves used in the expansion is controlled by the variable \verb|ecutwfc|. Similarly the number of waves used for computing the density is controlled by \verb|ecutrho| which by default is set to $4\verb|ecutwfc|$. In fig \ref{fig:bulk_ecut} one can see that with $ \verb|ecutwfc| = 80 \si{.Ry}$ error on the energy is of the order of $10^{-7} \si{.Ry}$. With this setting the effect of the cutoff on the density is negligible and so the default value is used.

    \begin{figure}
      \centering
      \resizebox{0.45\textwidth}{!}{\import{img/}{bulk_ecutwfc_rel2.pgf}}
      \resizebox{0.45\textwidth}{!}{\import{img/}{bulk_ecutrho_rel2.pgf}}
      \caption{Total energy and simulation time as a function of the two energy cutoffs.}
      \label{fig:bulk_ecut}
    \end{figure}

  \subsection{Results}
    The results of the scanning of the lattice parameter are reported in fig \ref{fig:bulk_a} while the results of the fit with with the Murnaghan equation are in tab \ref{tab:bulk_a}.

  \begin{figure}
    \centering
    \resizebox{0.45\textwidth}{!}{\import{img/}{bulk_a_fine.pgf}}
    \resizebox{0.45\textwidth}{!}{\import{img/}{bulk_a_fine_rel.pgf}}
    \caption{Behavior of the total energy and pressure as a function of the lattice constant for the non relativistic and relativistic pseudopotential respectively.}
    \label{fig:bulk_a}
  \end{figure}

  \begin{table}
    \centering
    \begin{tabular}{cccccccc}
      \toprule
        & $a_0\, [a.u.]$ & $K_0\, [\si{\kilo\bar}]$ & $K_0'$ \\
      \midrule
      non relativistic & $7.46834 \pm 0.00004$ & $841.0 \pm 0.2$ & $4.73 \pm 0.01$ \\
      relativistic & $7.624231 \pm 0.000009$ & $775.76 \pm 0.06$ & $4.995 \pm 0.006$ \\
      \midrule
      experimental \cite{rif:bulk_exp_data} & $7.646$ & $721$ & $4.72$ \\
      \bottomrule
    \end{tabular}
    \caption{Results of the fit of the pressure with the two pseudopotentials and corresponding experimental data}
    \label{tab:bulk_a}
  \end{table}














\begin{thebibliography}{100}
  \bibitem{rif:QE} P. Giannozzi et al. \emph{Quantum Espresso}  \url{http://www.quantum-espresso.org}
  \bibitem{rif:murnaghan} Wikipedia \emph{Murnaghan equation of state}  \url{https://en.wikipedia.org/wiki/Murnaghan_equation_of_state}
  \bibitem{rif:manual1}  \emph{Quantum ESPRESSO tutorial:  Self-Consistent Calculations,Supercells, Structural Optimization}  \url{http://indico.ictp.it/event/8301/session/95/contribution/528/material/slides/0.pdf}
  \bibitem{rif:bulk_exp_data} D.E. Grady \emph{Equatio of state for solids}  \url{https://doi.org/10.1063/1.3686399}
  \bibitem{rif:Al_relaxations} J.C. Zheng \emph{Multilayer  relaxation  of  the  Al(100)  and  Al(110)  surfaces:an  ab  initio  pseudopotential  study}  \url{https://www.sciencedirect.com/science/article/pii/S0368204800003625}
\end{thebibliography}

\end{document}
