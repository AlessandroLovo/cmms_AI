\documentclass[a4paper, 11pt]{article}
\usepackage[english]{babel}
\usepackage[top=2cm,bottom=2cm,left=2cm,right=2cm]{geometry}
\usepackage[utf8]{inputenc}
\usepackage{import}
\usepackage{float}
\usepackage{subfigure}
% \usepackage{subfig}
\usepackage[pdftex]{graphicx}
% \usepackage{graphicx}
\usepackage{amssymb,amsmath,amsthm,amsfonts}
\usepackage{xspace}
\usepackage{tabularx}
\usepackage{indentfirst}
\usepackage{wrapfig,booktabs}
%\usepackage[small]{caption}
% \usepackage{subcaption}
\usepackage{eucal}
\usepackage{eso-pic}
\usepackage{hyperref}
\usepackage{url}
\usepackage{booktabs}
\usepackage{afterpage}
\usepackage{parskip}
\usepackage{listings}
\usepackage{fancyhdr}
\usepackage{textcomp}
\usepackage{cite}
\usepackage{multirow,multicol}
\usepackage{setspace}
\usepackage[version=4]{mhchem}
\usepackage{nicefrac}
\usepackage{siunitx}

\usepackage{caption}
\captionsetup{tableposition=top,font=small,width=0.8\textwidth}
%\usepackage[table]{xcolor}
\usepackage[arrowdel]{physics}
\usepackage{mathtools}
\usepackage{tablefootnote}
\usepackage{enumitem}

\setlist[description]{font={\scshape}} %style=unboxed,style=nextline
\usepackage{floatflt}
\usepackage{commath}
\usepackage{bm}
\usepackage{ifthen}
\usepackage{comment}
\usepackage[colorinlistoftodos,textsize=tiny]{todonotes}

\newcommand{\overbar}[1]{\mkern 1.5mu\overline{\mkern-1.5mu#1\mkern-1.5mu}\mkern 1.5mu}
\let\oldfrac\frac
\renewcommand{\frac}[3][d]{\ifthenelse{\equal{#1}{d}}{\oldfrac{#2}{#3}}{\nicefrac{#2}{#3}}}


\begin{document}

\title{Study of the relaxation of the Al (100) surface via Ab Initio simulations}
\author{Alessandro Lovo, mat. 1236048}

\maketitle

\section{Introduction}
  The aim of this report is use Ab Initio (AI) methods to study the surface of an Aluminum crystal. In order to do this the Quantum Espresso \cite{rif:QE} will be used to first find the bulk value of the lattice constant for Al and then simulate a surface to study its relaxation from the bulk configuration.

\section{Estimation of the bulk lattice constant}
  The Al crystal has an fcc structure and the idea to find the bulk lattice constant $a$ is to simulate the crystal at different values of $a$ and for each of them compute the pressure $P$ resulting on the system. At this point data can be fitted with the Murnaghan equation of state:

  \begin{equation*}
    P(V) = \frac{K_0}{K_0'}\left(\left(\frac{V}{V_0} \right)^{-K_0'} - 1 \right)
  \end{equation*}

  where $V$ is the volume of the unit cell and $V_0$ the one at equilibrium, while the bulk modulus is $K = -V \left(\frac{\partial P}{\partial V}\right)_T \approx K_0 + K_0'P$
  \subsection{Optimization of the simulation parameters}






\begin{thebibliography}{100}
  \bibitem{rif:QE} P. Giannozzi et al. \emph{Quantum Espresso}  \url{http://www.quantum-espresso.org}
  \bibitem{rif:murnaghan} Wikipedia \emph{Murnaghan equation of state}  \url{https://en.wikipedia.org/wiki/Murnaghan_equation_of_state}
\end{thebibliography}

\end{document}
